\documentclass[11pt]{article}
\usepackage[utf8]{inputenc}
\usepackage{setspace}
\usepackage{lineno}
\usepackage{authblk}
\usepackage{natbib}
\usepackage[margin=2cm]{geometry}
\usepackage{tabularx}
\newcommand{\supersc}[1]{\ensuremath{^{\textrm{#1}}}}
\newcommand{\subsc}[1]{\ensuremath{_{\textrm{#1}}}}
\usepackage{booktabs}
\bibliographystyle{unsrtnat}
\usepackage{graphicx}
\newcommand{\quickwordcount}[1]{
		\begin{center}
			{\large Word count:
			\input{words.sum}}
		\end{center}
}


\title{\textbf{Modelling bacterial population growth and the effect of genus and medium and temperature of growth environment on model fit}}
\author[1]{Lizzie Bru}
\affil[1]{School of Life Sciences, Imperial College London, Silwood Park Campus, Ascot SL5 7PY, UK}
\date{}

\onehalfspacing
\linenumbers

\begin{document}
	
	\maketitle
	\quickwordcount
	
	\newpage
	
	\section{Abstract}
	
	1. Studying bacterial population growth is both important in many fields of biology, such as the study of infectious diseases, and interesting, since bacteria provide an excellent study system on which statistical modelling work can be done.
	\newline
	2. Here I will model bacterial population growth for multiple populations. I will specifically investigate which of the candidate models, two of which are mechanistic and two phenomenological, best fit to the growth curves, as well as whether genus and growth environment (specifically medium and temperature) predict the patterns of growth and therefore the best fitted model. 
	\newline
	3. I use Ordinary Least Squares (OLS) to fit the two phenomenological models (quadratic and cubic) and Non-linear Least Squares (NLLS) to fit the mechanistic models (logistic and Gompertz) to the growth curves. I then select the best model based on the Akaike Information Criterion (AIC) and analyse the effects of the predictors on the resultant best models of the growth curves.
	\newline 
	4. The majority of populations were best fitted by the Gompertz model, followed by logistic and cubic, and none were best fitted by the quadratic model. Genus, medium, and temperature of growth all affected which model was the best fit, with certain genera, media, and temperatures associated with particular growth patterns.
	\newline
	5. My findings suggest that mechanistic models may better explain bacterial population growth, and specifically that Gompertz model assumptions of carrying capacity limiting exponential growth and a time lag of the onset of population growth apply to many bacterial populations. Certain genera, media, and temperatures may furthermore shape the growth patterns of bacterial populations in many interesting ways. Hence modelling can help provide many interesting insights on bacterial population growth.
	
	
	\section{Introduction}
	
	Studying population growth rates is important in many fields in biology. Population demographics, particularly growth rates, help us to understand the size and status of populations and to forecast future population trends \cite{tarsi2012}. Quantifying growth rates is essential in a range of fields such as the conservation of endangered species, population management, and epidemiology. Fluctuations in population density are important for shaping dynamics and  emergent functional characteristics of ecosystems such as rates of carbon fixation and disease transmission (\cite{ewald1976toxicity, gao1992disease}).
	
	Many factors affect population growth rates. When abundance is low and resources are unlimited, theory predicts that populations should grow exponentially. If resources become limited, population growth should slow and eventually stop. There may additionally be a time lag before population growth takes off to begin with. Therefore population growth rates are affected by resource availability. Many other factors also influence growth rates due to the complexity of biological systems with multiple interacting components. Some of these factors include but are not limited to environmental stressors, population density, and survival and birth rates (\cite{sibly2002population, dinsmore2010assessment}).
	
	Studying bacterial growth rates in particular is both important and interesting. It is central in many fields. For example, studying bacterial virulence and antibiotic susceptibility is important in the study of infectious diseases, particularly facilitating our understanding and treatment of bacterial diseases (\cite{boulos2005molecular}). It is also important in helping inform the utilisation or inoculation of a bacterial isolate for use in producing antibiotics and other natural products, enhancing plant growth, and catalysing biodegradation of toxic organics. Bacterial population growth is also interesting from a statistical physics perspective, presenting many fascinating phenomena that pose new challenges to modelling (\cite{allen2018bacterial}). In batch culture, bacterial population growth generally follows a distinct set of phases. Firstly, the lag phase occurs whereby bacteria prepare for growth. This involves the activation of a suite of transcriptional machinery, including genes involved in nutrient uptake and metabolic changes. The subsequent exponential growth phase involves the division of bacteria at a constant rate, with the population doubling at each generation. Finally, bacterial populations reach a stationary phase when the carrying capacity of the growth medium is reached. During this phase, growth slows, with the number of cells stabilising.
	
	Traditional methods of measuring bacterial growth rates are limited. These involve plotting cell numbers or culture density against time on a semi-logistic graph and fitting a straight line through the exponential growth phase, with the slope of the line indicating maximum growth rate. However, bacterial growth trends are generally more complex, with this method failing to account for the whole sigmoidal growth curve, and are influenced by many parameters which are not taken into account in these methods. 
	
	More recently-developed modelling approaches provide a better way to describe bacterial growth. Modelling aims to establish the existence of statistically significant, non-random patterns in empirical data. Models are based on physical principles derived from expected behaviour of biological processes. This involves using biological knowledge to construct models, often involving trading off between realism, precision, and generality (\cite{levins1966strategy}). Since even the most flexible models have artificial assumptions, it is important to consider several competing models. Model selection replaces the traditional null hypothesis testing approach often used in science, whereby we ask whether we can reject a hypothesis in light of observed data (\cite{johnson2004model}). It is grounded in likelihood theory, whereby models are ranked and weighted to obtain a quantitative measure of relative support for each. Model selection involves the following steps: generating biological hypotheses as candidate models; fitting the models to the data; selecting the best model, for instance using criteria such as Akaike's information criterion (AIC) (\cite{akaike1998information}) or the Schwarz criterion (also known as the Bayeseian information criterion (BIC) (\cite{schwarz1978estimating})); and finally using the fitted model(s) to estimate parameters to enable us to make predictions (\cite{johnson2004model}).	
	
	The two principal classes of models - mechanistic and phenomenological - both play a key role in biology. Phenomenological models such as polynomial models reveal patterns in data that generate hypotheses which can be tested using further model fitting. Mechanistic models are based on population growth theory. They aim to generate more accurate, mechanistic hypotheses about the mechanisms underlying patterns in the data. They are useful in providing the starting point to a bottom-up approach to understanding ecosystem dynamics; mechanisms of individual dynamics can help us to characterise mechanisms of interactions, which can then aid our understanding of mechanisms of community dynamics.
	
	The relative merits of phenomenological and mechanistic models forms an interesting topic of discussion in the field of population biology modelling. While mechanistic models are ultimately, when empirically-grounded and successful, the best path towards a theory, phenomenological models are often easier to fit to data. These latter models often perform better than mechanistic models because they generally have less restrictive assumptions. Most ecological studies perform phenomenological modelling due to our need to forecast future population trends rather than explain the mechanisms underlying them. However, this then leads to the important question of whether we can legitimately forecast without explaining.

	This study's aims are two-fold. Firstly, I will examine how well different models - two phenomenological (cubic and quadratic polynomial) and two mechanistic (logistic and Gompertz (\cite{gompertz1825xxiv})) - fit to growth curves in various bacterial populations. I will then test for whether the genus of the bacteria, as well as the medium and temperature in which the population was grown, affect which model fit best to the population growth curve, in this way examining the factors influencing population growth rates. I will hence illustrate how simple modelling and testing for predictor effects can help us to characterise bacterial population growth.

	
	\section{Methods}
	
	\subsection{Data}
	
	I used data from ten different studies which investigated bacterial population growth (\cite{roth1962continuity, stannard1985temperature, phillips1987relation, sivonen1990effects, gill1991growth, zwietering1994modeling, bae2014growth, galarz2016predicting, bernhardt2018metabolic, silva2018modelling}). These data include population or biomass measurements at corresponding time intervals and the temperature, species, and medium used for the population growth in the experiment.
	
	I subsetted the data into groups of individual growth curves based on species of the bacterial population, the temperature and medium at which populations were grown, and the citation of the study which collected the data. This yielded 285 subsets, each with one bacterial growth curve. I performed all subsequent model fitting and analyses using these subsets.
	
	\subsection{Computing tools}
	
	I used Python version 3.8.10 (\cite{10.5555/1593511}), specifically using the packages Pandas (\cite{mckinney2010data}) and NumPy (\cite{harris2020array}) for data wrangling since Pandas data frames are easy to manipulate.
	
	I used R version 3.6.3 (\cite{R_citation}), specifically using the packages ggplot2 (\cite{ggplot2_citation}), minpack.lm (\cite{elzhov2010r}), qpcR (\cite{Ahmed2018-ni}), viridis (\cite{viridis_citation}), ggpubr (\cite{kassambara2020package}), (olsrr \cite{hebbali2017package}), and dplyr (\cite{wickham2015dplyr}) for model fitting, plotting, comparisons, and statistical analyses. I did this because R has numerous packages such as these for data processing and visualising, notably those such as ggplot2 which permit excellent graphing and visualisation. 
	
	For writing up the report, I used \LaTeX since it produces well-formatted documents with easy-to-embed images.
	
	Finally, I compiled the project into a reproducible workflow, including converting the report from \LaTeX to pdf, using bash version 5.0.17 (\cite{gnu2007free}) since this language provides a simple way to automate the compilation of multiple scripts.
	
	
	\subsection{Model fitting}
	
	I used Ordinary Least Squares (OLS) to fit the quadratic (Eq. (1)) and cubic (Eq. (2)) models to each growth curve and Non-Linear Least Squares (NLLS) to fit the logistic (Eq. (3)) and Gompertz (Eq. (4)) (\cite{gompertz1825xxiv}) models.
	
	\begin{equation}
		N_t = at^2 + bt + c
	\end{equation}
	
	\begin{equation}
		N_t = at^3 + bt^2 + ct + d
	\end{equation}
	
	\begin{equation}
		N_t = \frac{N_0Ke^{rt}}{K + N_0(e^{rt} - 1)}
	\end{equation}
	
	\begin{equation}
		log(N_t) = N_0 + (N_{max} - N_0)e^{-e^{r_{max}exp(1)\frac{t_{lag}-t}{(N_{max}-N_0)log(10)}+1}}
	\end{equation}
	
	Where population density at time \emph{t} depends on the various parameters in each equation: initial population density ($N_0$), carrying capacity ($K$), growth rate ($r$), maximum population density ($N_{max}$), the duration of the delay before the population starts growing exponentially ($t_{lag}$), and maximum growth rate ($r_{max}$).
	
	I only fitted models to subsets which contained four or more data points so that this was greater than the number of parameters in the most complex model (Gompertz). 
	
	For the logistic model, I assigned the smallest population size as the starting value for population size ($N_0$), the highest population size multiplied by two as the carrying capacity ($K$) starting value (multiplying it by two increased the proportion of model convergence), and set the starting value for $r_{max}$ as an arbitrary low value of $10^{-8}$ since this lead to better model fitting than my original attempt of setting the $r_{max}$ starting value as the maximum population in the curve divided by the time-step.
	
	To satisfactorily fit the Gompertz model to as many datasets as possible, I sampled each of the starting values randomly simultaneously. Since I had reasonably high confidence in the means, I sampled the starting values for $N_0$, $K$, and $t_{lag}$ from a normal distribution with the mean for $N_0$ as the smallest population size, that for $K$ as two times the largest population size, and the mean for $t_lag$ as the last timepoint of the lag phase, with a standard deviation of three times this for both. Since I used an arbitrarily small mean for $r_{max}$ similarly to for the logistic model since this maximised the number of datasets which this model fit to. I therefore had lower confidence in the mean, so sampled the starting values for $r_{max}$ from a uniform distribution. I arbitrarily set lower and upper bounds to be $10^{-10}$ and $10^{-2}$ since these maximised the number of datasets which the model successfully fit to.
	
	
	\subsection{Model comparison}
	
	For each model, I calculated Akaike's information criterion (AIC, Eq. (5)) as the fit statistic. 
	
	\begin{equation}
		AIC = -2.ln(L) + 2p
	\end{equation}
	
	For each subset, I selected the best model out of the four fitted based on AIC: the best model is the lowest model which differs from the second lowest model by more than two AIC units.
	
	To examine relative abilities of each of these models to fit to the growth data, I calculated the proportions of which model was selected as the best model across all the subsets for which there was a best model. 
	
	
	\subsection{Statistical analysis of predictors of best models}
	
	Using the dataset of each subset with its corresponding best model, I tested for the effects of genus of the bacteria concerned, and temperature and medium under which it was grown, on which model was the best for this subset using an intrinsic, two-sample Chi-square tests.
	
	
	
	\section{Results}
	
	\subsection{Data}
	
	I discarded anomalous or unexplained datapoints such as those associated with negative population density and time values. The final dataset contained 285 subsets of bacterial growth curve.
	
	
	\subsection{Model fitting}
	
	I fitted as many of the four models (quadratic, cubic, logistic, and Gompertz) as possible to each individual growth curve, with an example of one growth curve and the fitted models shown in Figure 1. Quadratic, cubic, and logistic models fitted satisfactorily to all 285 growth curves, and the Gompertz model fitted to 283 of the 285 curves.  

	\begin{figure}[htbp]
		\centering
		\includegraphics[scale=1.3]{../results/fig1.png}
		\caption{Population growth curves fitted by quadratic, cubic, logistic, and Gompertz models for the growth of \textit{Aerobic Mesophilic} under 4\supersc{o}C in raw chicken breast, measured by \cite{bae2014growth}.}
		\label{fig1}
	\end{figure}
	
	
	\subsection{Model comparison}
	
	For 232 of the growth curves, comparisons of AIC values indicated that one of the four models provided a significantly better fit than the others. For the remaining 54 growth curves, there was no model which provided a significantly best fit. I used only the 232 growth curves for which there was clearly a best model in subsequent analyses. 
	
	Gompertz was the best model for the largest proportion of curves, followed by logistic, cubic, and quadratic, which was not the best model for any of the curves (Table 1).

	\begin{table}[htbp] 
		\caption{Count and proportions of each model being the best fit for bacterial populations.}
		\centering      % center table 
		\begin{tabular}{c c c}  % center columns (3 columns) 
			\hline                        %horizontal line
			Model & Count & Proportion \\ [0.5ex] % adds vertical space
			\hline                    % inserts single horizontal line 
			Gompertz & 145 & 0.628  \\
			Logistic & 63 & 0.273 \\ 
			Cubic & 24 & 0.104 \\ 
			Quadratic & 0 & 0.000 \\ 
			 [0.5ex]
			\hline     %inserts single line 
		\end{tabular} 
		\label{table1}  % is used to refer this table in the text 
	\end{table} 

	
	\subsection{Predictors of the best model}
	
	The genus of the bacteria ($\chi^2$ (42, N = 232) = 188.16, p < 0.001; Figure 2A; Table 2), as well as the medium ($\chi^2$ (34, N = 232) = 196.83, p < 0.001; Figure 2B; Table 3) and temperature ($\chi^2$ (32, N = 232) = 127.84, p < 0.001; Figure 2C; Table 4) they were grown in, significantly affected which model best described the bacterial growth trend. In particular, bacterial populations which showed growth curves best described by the Gompertz model were predominantly of the genera Arthrobacter and Aerobic; those best described by the logistic model were mostly Acinetobacter and Stenotrophomonas; and those best fitted by a cubic model were mostly Serratia (Table 2A, B, C). Bacterial populations grown on TSB mostly showed logistic growth curves, in fact accounting for all populations best fitted by this curve (Table 3B). Similarly, those best fitted by the Gompertz model were biased towards TGE agar and those by a cubic model towards vacuum beef striploins and cooked chicken breast, although there was greater variation in media for populations best fitted by these two models (Table 3A, C). Less clear trends are visible for temperature, although bacterial populations best fitted by the logistic model were mostly grown under higher temperatures with the exception of those grown under five\supersc{o}C, and populations best explained by the Gompertz model were slightly biased towards being grown at 22\supersc{o}C (Table 4A, B, C).
	
		\begin{figure}[htbp]
		\centering
		\includegraphics[height=18cm, width=18cm]{../results/fig2.png}
		\caption{Bar plots showing how genus, temperature, and medium affect which model fitted the bacterial growth curves the best.}
		\label{fig2}
	\end{figure}

	
	\begin{table}[htbp]
		\caption{Frequency of genera of the bacterial populations best fitted by (A) cubic, (B) logistic, and (C) Gompertz models.}
		\label{table2}
		\begin{tabularx}{\linewidth}{l*{6}{c}}
			\toprule
			\multicolumn{7}{l}{\textbf{Panel A: Cubic model}} \\
			\midrule
			\textbf{Genus} & \textbf{Frequency} \\
			Serratia & 6 \\
			Pseudomonas & 5 \\
			Aerobic & 4 \\
			Spoilage$^
			{1}$ & 3 \\
			Clavibacter & 2 \\
			Salmonella & 2 \\
			Escherichia & 1 \\
			Oscillatoria & 1 \\
			\bottomrule   
		\end{tabularx}
		\begin{tabularx}{\linewidth}{l*{7}{c}}
			\toprule
			\multicolumn{7}{l}{\textbf{Panel B: Logistic model}} \\
			\midrule
			\textbf{Genus} & \textbf{Frequency} \\
			Acinobacter & 12 \\
			Stenotrophomonas & 10 \\
			Bacillus & 8 \\
			Pantoea & 8 \\
			Pseudomonas & 8 \\
			Dickeya & 5 \\
			Klebsiella & 4 \\
			Chryseobacterium & 3 \\
			Clavibacter & 2 \\
			Pectobacterium & 2 \\
			Enterobacter & 1 \\
			\bottomrule
		\end{tabularx}
			\begin{tabularx}{\linewidth}{l*{7}{c}}
		\toprule
		\multicolumn{7}{l}{\textbf{Panel C: Gompertz model}} \\
					\midrule
		\textbf{Genus} & \textbf{Frequency} \\
		Arthrobacter & 29 \\
		Aerobic & 24 \\
		Pseudomonas & 14 \\
		Lactobacillus & 13 \\
		Staphylococcus & 10 \\
		Serratia & 7 \\
		Escherichia & 5 \\
		Salmonella & 5 \\
		Spoilage & 5 \\
		Tetraselmis & 5 \\
		Clavibacter & 4 \\
		Klebsiella & 4 \\
		Pantoea & 4 \\
		Weissella & 4 \\
		Dickeya & 3 \\
		Enterobacter & 3 \\
		Oscillatoria & 2 \\
		Pectobacterium & 2 \\
		Chryseobacterium & 1 \\
		Stenotrophomonas & 1 \\
		\bottomrule
	\end{tabularx}
		
		[1]\ This refers to a combination of different bacterial strains which cause food spoilage.
	\end{table}
	
	
		\begin{table}[htbp]
		\caption{Frequency of media in which the bacterial populations best fitted by (A) cubic, (B) logistic, and (C) Gompertz models were grown.}
		\label{table3}
		\begin{tabularx}{\linewidth}{l*{6}{c}}
			\toprule
			\multicolumn{7}{l}{\textbf{Panel A: Cubic model}} \\
			\midrule
			\textbf{Medium} & \textbf{Frequency} \\
			Vacuum beef striploins & 5 \\
			Cooked chicken breast & 4 \\
			APT broth & 2 \\
			Raw chicken breast & 2 \\
			TSB & 2 \\
			UHT skim milk & 2 \\
			CO\subsc{2} beef striploins & 1 \\
			Pasteurised double cream & 1 \\
			Pasteurised skim milk & 1 \\
			Salted chicken breast & 1 \\
			UHT double cream & 1 \\
			UHT full-fat milk & 1 \\
			Z8 & 1 \\
			\bottomrule   
		\end{tabularx}
		\begin{tabularx}{\linewidth}{l*{7}{c}}
			\toprule
			\multicolumn{7}{l}{\textbf{Panel B: Logistic model}} \\
			\midrule
			\textbf{Medium} & \textbf{Frequency} \\
			TSB & 63 \\
			\bottomrule
		\end{tabularx}
		\begin{tabularx}{\linewidth}{l*{7}{c}}
			\toprule
			\multicolumn{7}{l}{\textbf{Panel C: Gompertz model}} \\
			\midrule
			\textbf{Medium} & \textbf{Frequency} \\
			TGE agar & 29 \\
			TSB & 22 \\
			Salted chicken breast & 18 \\
			Raw chicken breast & 16 \\
			Cooked chicken breast & 13 \\
			MRS broth & 13 \\
			Vacuum beef striploins & 9 \\
			CO\subsc{2} beef striploins & 6 \\
			ESAW & 5 \\
			MRS & 4 \\
			UHT full-fat milk & 2 \\
			Z8 & 2 \\
			APT broth & 1 \\
			Pasteurised double cream & 1 \\
			Pasteurised full-fat milk & 1 \\
			Pasteurised skim milk & 1 \\
			UHT double cream & 1 \\
			UHT skim milk & 1 \\
			\bottomrule
		\end{tabularx}
	\end{table}
	
			\begin{table}[htbp]
		\caption{Frequency of temperatures in which the bacterial populations best fitted by (A) cubic, (B) logistic, and (C) Gompertz models were grown.}
		\label{table4}
		\begin{tabularx}{\linewidth}{l*{6}{c}}
			\toprule
			\multicolumn{7}{l}{\textbf{Panel A: Cubic model}} \\
			\midrule
			\textbf{Temperature (\supersc{o}C)} & \textbf{Frequency} \\
			20 & 5 \\
			10 & 4 \\
			15 & 3 \\
			4 & 2 \\
			6 & 2 \\
			7 & 2 \\
			8 & 2 \\
			30 & 2 \\
			2 & 1 \\
			25 & 1 \\
			\bottomrule   
		\end{tabularx}
		\begin{tabularx}{\linewidth}{l*{7}{c}}
			\toprule
			\multicolumn{7}{l}{\textbf{Panel B: Logistic model}} \\
			\midrule
			\textbf{Temperature (\supersc{o}C)} & \textbf{Frequency} \\
			35 & 18 \\
			5 & 16 \\
			15 & 16 \\
			25 & 16 \\
			\bottomrule
		\end{tabularx}
		\begin{tabularx}{\linewidth}{l*{7}{c}}
			\toprule
			\multicolumn{7}{l}{\textbf{Panel C: Gompertz model}} \\
			\midrule
			\textbf{Temperature (\supersc{o}C)} & \textbf{Frequency} \\
			15 & 22 \\
			7 & 16 \\
			20 & 15 \\
			10 & 14 \\
			25 & 11 \\
			30 & 11 \\
			2 & 9 \\
			4 & 9 \\
			5 & 6 \\
			12 & 6 \\
			0 & 5 \\
			8 & 5 \\
			16 & 4 \\
			35 & 4 \\
			37 & 4 \\
			6 & 3 \\
			32 & 1 \\
			\bottomrule
		\end{tabularx}
	\end{table}

	
	\newpage
	
	\section{Discussion}
	
	In this study I have shown that Gompertz and logistic models better explained the growth curves of the bacterial populations studied than the phenomenological cubic and quadratic models did. Which of these models fit best to each population was additionally affected by the genus of the bacteria in question as well as the medium and temperature in which the bacteria were grown.
	
	These findings firstly suggest that mechanistic models may better explain bacterial growth than phenomenological models do. This is relatively unexpected since phenomenological models often perform better than mechanistic models due to their generally less restrictive assumptions. This could suggest some interesting conclusions about modelling bacterial population growth. Logistic models include a way to factor in the negative feedback effect of populations reaching the carrying capacity of their environment since it factors in that exponential population growth levels off when resources decline and start to become limiting. It is therefore perhaps unsurprising that a logistic model better explains the bacterial population growths in these data; the bacteria were grown in finite media so carrying capacity will inevitably be reached. Therefore, simple polynomial models which fail to capture this levelling off of population growth should provide worse fits to the data. The Gompertz model then goes one step further than the logistic model. A logistic model can show deviations from the data due to its assumption that the population is growing from the very beginning; in reality the population often takes some time before it starts growing exponentially. This is because bacteria often take some time to acclimate to a fresh growth medium, new resource, or new environment which they are grown in. The Gompertz model better captures this lag phase than the logistic model does by incorporating the extra $t_{lag}$ parameter. Hence the fact that the Gompertz model was the best model for the majority of bacterial growth curves suggests that this time lag and carrying capacity limitation occurs in many populations, and therefore that the mechanisms of bacterial population growth often involve these two components.  
	
	My findings also suggest the importance of bacterial genus and growth environment in shaping population growth patterns. This builds on much previous work which has highlighted the variety of factors affecting population growth trajectories (\cite{sibly2002population, dinsmore2010assessment}). My findings are particularly useful in this regard in providing simple but clear suggestions of how genus and growth environment affect population growth since comparisons between these models are very straightforward. In particular, where certain the growth patterns of certain genera were best fit by a cubic model, this suggests that carrying capacity limiting exponential growth and time lags of the onset of growth are not especially defining features of growth. Contrastingly, genera whose growth was best fitted by a logistic model, such as \textit{Acinobacter} and \textit{Stenotrophomonas}, are likely to suffer particularly strongly from limiting resources slowing their population growth. Similarly, those whose growth was best fitted by the Gompertz model, notably \textit{Arthrobacter} and \textit{Aerobic}, are likely to show population growth decrease due to carrying capacity limitations as well as a time lag of onset of growth as they take longer to acclimate to novel media, for instance. The effect of medium also yields similar, interesting conclusions. The strong association of the TSB medium with logistic growth patterns suggests that this medium becomes strongly limiting as populations grow (perhaps by being used up faster than other media), but also that bacteria perhaps take little time to acclimate to this medium, since otherwise the Gompertz model would have provided better fit. Media associated with Gompertz patterns of growth may conversely be associated with becoming limited as populations approach carrying capacity as well as bacteria taking more time to acclimate to them, causing the time lag captured by the Gompertz model. Similar conclusions apply to the temperature of the growth environment. For example, higher temperatures appear to be broadly associated with logistic growth patterns, suggesting that at higher temperatures bacteria take little time to acclimate to the growth environment but carrying capacity limits exponential growth.
	
	There is much interesting work which could build well on this study. Firstly, future work could use more data to avoid biases. The dataset which I was provided with contains many biases, as expected since the data were gathered from a range of sources and not collected specifically for this study. For example, there are biases towards certain genera, with \textit{Aerobic},\textit{Arthrobacter}, and \textit{Pseudomonas} species making up 12.6\%, 12.3\%, and 11.2\% of the data, respectively, even though there are 22 genera in total. This could lead to issues surrounding conclusions; for example, is the Gompertz model the most popular simply because the dataset contains more genera or growth environments which are associated with carrying capacity limitations and time lags of initial growth. With more time it would also have been interesting to average competing best models for the subsets which did not clearly have a best model. This would be particularly helpful if parameters derived from models were then used to make predictions about bacterial growth; an interesting future avenue. Future work could also investigate more potential predictors of population growth rate, for instance sex ratio, as well as other candidate models. Potential models include stochastic differential equation models, which \cite{alonso2014modeling} showed to successfully explain bacterial population growth including the time lag phase, and Monod's model (\cite{lobry1992monod}), for example. Future work could additionally explore other methods to select the best model, for instance using BIC or Free Energy, which (\cite{penny2012comparing}) found to be a better model comparison criterion than BIC and AIC for certain models.
	
	
	Importantly, future work should investigate potential confounding effects of collinearities between predictors. For example, the fact that the TSB medium was strongly associated with logistic growth patterns could have been confounded by the particular genus or temperature associated with that medium in my dataset. With more time, interesting statistical analysis to account for this potential confounding would be useful to help draw more robust conclusions about which factors predict model fitting to population growth.
	
	My findings therefore open up many interesting avenues for future research. Improving our understanding of bacterial population growth is highly important both in biological applications such as the use of bacterial cultures in industry, as well as in the field of statistical modelling. Ongoing work to investigate how to model bacterial population growth as well as which factors might predict growth patterns is therefore both timely and fascinating.
	
	\section{Acknowledgements}
	
	With thanks to \cite{roth1962continuity, stannard1985temperature, phillips1987relation, sivonen1990effects, gill1991growth, zwietering1994modeling, bae2014growth, galarz2016predicting, bernhardt2018metabolic, silva2018modelling} for supplying the data used in this study. I additionally would like to thank Dr Alex Christensen, the teaching assistants for the Computational Methods in Ecology and Evolution (CMEE) course at Imperial College London, and other students in the CMEE course for their support and advice throughout the scripting and write-up process.
	
	\bibliography{miniprojwriteup}
	
\end{document}