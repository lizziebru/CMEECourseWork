\documentclass[9pt]{article}
\usepackage[utf8]{inputenc}
\usepackage{setspace}
\usepackage{lineno}
\usepackage{authblk}
\usepackage{natbib}
\usepackage[margin=1cm]{geometry}
\usepackage{tabularx}
\newcommand{\supersc}[1]{\ensuremath{^{\textrm{#1}}}}
\newcommand{\subsc}[1]{\ensuremath{_{\textrm{#1}}}}
\usepackage{booktabs}
\bibliographystyle{unsrtnat}
\usepackage{graphicx}

\title{\textbf{Seminary diary}}
\author[1]{Lizzie Bru}
\affil[1]{School of Life Sciences, Imperial College London, Silwood Park Campus, Ascot SL5 7PY, UK}
\date{}

\onehalfspacing


\begin{document}
	
	\maketitle
	
	\newpage
	
	\section{\textit{Underground forests, fire, mammal browsers, and the origins of African savanna}. Professor Jonathan Davies, University of British Columbia}
	
	Major question: what allowed savanna to expand and separated it from other biomes?
	
	Background on biomes in South Africa: Fynbos are one of the many hyperdiverse biomes, being both very small and very diverse. Contrastingly to Fynbos, the Namib desert biome has very low diversity, but contains some of the tallest sand dunes in the world. The grassland biome resembles savanna landscapes but has a different community of flora. The Indian Ocean Coastal Belt biome is more forested. The Miombo woodland biome contains more thickets and shrub-like vegetation, and is the closest biome to the savanna biome. The Savanna biome is present across much of South Africa. It contains sporadic trees interspersed with grassland wherever they are not dense enough to competitively exclude grass. Savanna could support forests, however, so there must be something preventing forests from dominating this biome.
	
	Phylogenetic analysis: DNA barcoding of the Trees of Africa (initiative run by the African Centre for DNA barcoding) was used to build a phylogenetic history of South African flora. They sequenced anything that fills a tree-like role in the landscape, randing from shrub to bamboo to climbers. Daru et al. (2015) used the tree to generate a new map of the biomes of South Africa based on phylogenetic relationships. This showed that these biomes contain species that phylogenetically cluster together.
	
	Origin of these biomes: could this phylogenetic information be used to address the origin of these biomes? Professor Davies and his colleagues successively removed phylogenetic structure to identify the evolutionary depth at which the biomes collapse to see how the biomes collapse into each other. Their initial best guess for the age of savanna is 8-60 million years. While climate is the usual explanation for changing vegetation, it does not match the time of savanna spread. They found that there was a decrease in carbon dioxide levels and C4 grass lineages; while this does not on its own explain the origin of savannas, it was perhaps necessary. A kickstart was therefore needed for the ecological expansion into savanna. A possibility is fire which would have pushed back the forest. Grazers could also push back the forest. These two competing explanations are both good possibilities, but there is little evidence supporting either. We therefore have a very poor record of what the ultimate drivers of savanna origin were due to no information on browser or fire intensity from the fossil record. On possibility is studying the "underground forests" of Africa (\textit{Geoxylic suffrutex}): unique species where most of them are underground, but their sister lineages look like trees. They are only found in savanna grassland, so their age could be useful for dating savanna since they require savanna to exist. They could be used as markers for fire since their growth evolved in response to frequent fires and high precipitation. Therefore, perhaps fire was one of the ecological forces that allowed savanna expansion. They performed Bayesian reconstruction of the divergence times between two of these sister taxa (\textit{Cussonia arborea} and \textit{Cussonia corbisieri}). This showed that the radiation of underground trees happened multiple times independently. Despite large uncertainty with ages, the results showed that there are no very old underground trees, with most being 2-4 million years old and no evidence of old underground trees at all in the south. Perhaps they therefore originated at the equator and more recently evolved in the south, suggesting that fire drove the expansion of savannas, originating at the equator and driving south. 

	
	
	\section{\textit{The limits to ecological limits to diversification}. Professor Rampal Etienne, University of Groningen}
	
	Introduces the competition-related hypothesis, which hypothesises that more closely related species will share similar habitats and constitutions and thus experience higher competition and speciation rates than unrelated species do. There is evidence of limits to ecological diversification. Fossils of extinct and current diversity tell us that extinction happens/diversity does not always increase exponentially, providing evidence for biodiversity becoming limited at the species level/macroevolutionary scale. Molecular phylogenies of extant species show that diversity does not increase exponentially. Both fossils and phylogenies provide some evidence suggestive of negative feedback of diversity on diversification, suggesting that diversification is diversity-dependent. Possible biological mechanisms include: niches become increasingly occupied and eventually limit diversification; or niche construction slows down as species abundance increases (although in some cases it can go up too). The problem is that many phylogenies don't plateau or show an inverted S-shape. Possibly not enough time was allowed. The model may therefore be inaccurate (e.g. protracted speciation instead could explain the slow-downs, we could be over-simplifying adaptive radiation). This suggests that whether diversification depends on diversity depends on the species in question. If this is the case, this leads to the question of what are these limits to diversity-dependence i.e. the limits to ecological limits to diversification. Islands could be a good way to answer this question. 
	
	There are two types of diversity-dependence: clade-specific diversity-dependence, which only applies to diversity within the clade established by the mainland ancestor; and island-wide diversity-dependence, where all species on the island affect each other. They present a new method for distinguishing these two types of diversity-dependence. This involves using the stochastic framework Dynamic Assembly of Island biotas through Speciation, Immigration and Extinction (DAISIE). This allows diversity-dependent diversification and immigration, and parameters estimated using likelihood of phylogenetic data can be used for fully stochastic simulations. (Remember rates of colonisation and extinction depend on no. of species present on an island - theory of island biogeography). Simulation work showed that DAISIE is robust to model violations: specifically time-varying rates due to island ontogeny/environmental fluctuations, trait-dependent speciation, extinction on the mainland, mainland sampling, and non-oceanic island origin. They used DAISIE to estimate rates of extinction, using parametric bootstrapping. Using DAISIE on Hispaniola rain frogs, they found evidence for clade-specific but not island-wide diversity-dependence. The observed repeated patterns could suggest that niche differences exist between different clades. To summarise, they therefore found evidence for limits to ecological limits to diversification using tools to detect this from phylogenetic data, notably from islands. Going forwards, they therefore developed a method which is operational, although it is mathematically and computationally challenging, and there are not many datasets currently available for larger-scale hypothesis testing.


	\section{\textit{Birdwatching on a cosmic scale}. Dr Daniel J. Field, University of Cambridge}
	
	Dan's work is based on understanding how the diversity of birds has arisen. Over the last 10-15 years, this has become really important in ornithology, and major steps have been made in resolving evolutionary relationships among living groups of birds (e.g., \cite{prum2015comprehensive}). A particularly important point in time is the end Cretaceous mass extinction, when a major asteroid (11-81km in diameter, travelling at ~20km/s, causing an impact crater fo 150km in diameter and 20km deep) struck the Earth. It drove major groups of organisms to extinction, but also it's important to consider how this event may have paved the way for structuring survivorship/diversification patterns for those lineages that survived the event. We thus want to understand how this major event played a role in the evolutionary diversification of birds. 
	
	Gave a quick summary of how the K-Ph transition influenced bird evolution (from \cite{longrich2011mass} and other work by Dan mostly). Loads of major groups of organisms disappeared. We knew there were many lineages of these birds in the Cretaceous but didn't know when they went extinct. All of these major lineages extend through the fossil record up until this asteroid impact then they don't appear in the fossil record anymore, suggesting their extinction was probably driven by this impact. But we know that birds exist in the present day, so some of them must have survived. So Dan's research showed that birds suffered a lot in this mass extinction event, but remaining questions are: which birds did survive? How did they manage to survive? What happened to bird habitats?
	
	To answer the first question of which birds survived, they used a time-calibrated phylogeny. They think that only a small handful of crownbird lineages survived this mass extinction event (\cite{longrich2011mass}). To answer the second question about how these birds survived, they looked into body size because he thinks it was a highly size-selective event: hardly anything larger than 5kg survived. And they found evidence for this: relatively strong evidence that post-extinction crownbirds were smaller than pre-extinction ones (\cite{berv2018genomic}). For the question of what happened to bird habitats: they hypothesised that global deforestation occurred as a result of the impact inducing a selective filter against arboreal birds. They tested this by incorporating ecological data into large-scale ancestral state reconstruction. The Mesozoic fossil record is relatively rich for providing evidence of what pre-modern birds were like in the late part of the Cretaceous. They found that only ground-dwelling lineages survived the mass extinction event (\cite{field2018early}).
	
	Three key hypotheses about crownbird macro-evolution following this mass extinction event: deep evolutionary branch (only the deepest lineages survived), relatively small-bodied (any surviving lineages probably would've been relatively small-bodied), and non-tree-dwelling (any surviving lineages probably would've been mostly non-tree-dwelling). But these are only hypotheses: the only way to test them is to look at the crownbird fossil record. However, the earliest fossils of crownbirds in the Cretaceous is very scarce. They did find this one fossil encased in limestone: an unassuming holy grail (\cite{field2020late})! It's a fossil of a femur and tibiotarsus. Dates back to just before the asteroid impact. CT-scanned it and found it has a nearly complete 3D skull. Lower jaw is fused together, lots of features support the idea that it's a member of the Neornithes clade (early modern-type bird) - they gave it a new taxon name, Galliforme - , and phylogenetic analyses support this too. What does this fossil tell us about these hypotheses? Hypothesis 1: it's one of the 3 deepest lineages in the crownbirds - and it crossed the extinction event. Hypothesis 2: it's considerably smaller than many birds from that time period that were also ground-dwelling. Hypothesis 3: hind-limb proportions suggest it was ground-dwelling. So this fossil supports these hypotheses! Plus it's from Europe which has implications for future more European finds!


	\section{\textit{A call to ARMS as a Marine Biodiversity Observation Network}. Dr Christopher Meyer, Dr Sarah Tweedt, National Museum of Natural History, Smithsonian Institution}
	
	Chris: works on assessing the vast crypto-benthic community diversity in French Polynesia. Big question: how should we count and inventory? How (e.g. DNA) and where should we measure? This is how the ARMS project came about. They stack PVC plates laid on the seafloor then recover them. These are standardised everywhere, so they use these as a standard metric of biodiversity. They do size fractionation so that not only one creature dominates. Their data shows remarkable turnover across the Pacific Ocean. Using standardised transects, they found that you can only find one percent of all the species everywhere. How can they use ARMS to study how diversity responds to change? What are the tipping points for communities? They collected data across  gradients of degradation/recovery: anthropogenic gradients (more human impact == lower diversity), pH gradients (to look at effects of ocean acidification: found lower diversity when lower pH), depth gradients (== deeper spots have less diversity than shallower spots), protected vs unprotected areas (found that systems are able to recover pretty well - regain higher diversity within 2-3 years when area is put under protection). They also tested this with different marker choices: found that overall you get pretty similar results. They found that the year-to-year effect is more than the space-to-space effect. They want to expand to a global ARMS program, to make it a tool that anyone can use throughout the world. Also useful as an invasive species detector. It's now been adopted in the newly formed European Marinomics network, with 20 ARMS observatories, 14 European countries, and half of the sites being used to monitor for non-indigenous species. 
	
	Sarah: ARMS are great to sample marine sponges: they sample cecil-encrusting organisms. Sponges have had ~800 million years to evolve, and most arms plates collect loads of taxonomically-important sponges. eDNA (= DNA collected from environmental samples (soil, water etc))is really useful for ecosystem monitoring, species detection etc. And it's cheap and easy to implement. It's a hot topic at the moment, with government agencies asking how we can use eDNA to monitor things more cheaply etc. ARMS as a teabag/seepage experiment: can eDNA from ARMS-bin water samples recover the ARMS profile? Will sponges be the first organisms detected in eDNA from ARMS water? Relevant paper: \cite{de2009cell}.They chose 2 ARMS to soak in their seepage experiment. Sampling protocol: 2 Tetiaroa ARMS (B ad D); first sample = time 0 = ambient water from locality; collect ARMS water for 8 time points, 0-24 hours; water in bin gently mixed, 1.0 litres collected, 40 micrometre-filtered to remove plankton particles; later filtered through 0.22 micrometre sterivex (50ml syringe), preserved in ~2.0ml ATL. Extraction, amplification, and sequencing protocol: digestion and DNA extraction; COI amplification with index adaptors, indexing PCR; normalize, pool samples, MiSeq sequencing. Sequence analysis protocol: import raw sequences + trim primers; read quality control; generate 'ASV' table + analysis. Gave some details about each step (although gave a bit too much jargon-y detail on this for a talk). Cool result: they found that eDNA and ARMS separate nearer the bottom. Found lots on marine fungi. Also, two species of parrotfish (+ also diadema) spawned and every single sample is swamped with parrotfish and diadema DNA: this really shifted the eDNA profiles. More recent work: \cite{antich2021marine}. Most reads in the ARMS get picked up in the eDNA which is good. 
	
	
	
	\section{\textit{Genetic and epigenetic variation in mycorrhizal fungi and its role in altering plant growth and carbon storage in tropical soils}. Dr Ian Sanders, University of Lausanne}
	
	\textit{Introduction}. Malnutrition kills loads of people annually. What can scientists do about it? Cassava! It's very important for food security in the face of climate change. Cassava has a relatively stable yield during perturbation of climate (better than other crops). How can we tackle malnutrition with fungi? Arbuscular mycorrhizal fungi (hereafter AMF) helps plants grow bigger by helping plants get more nutrition from the soil. Lost of people work trying to improve cassava production: cassava responds very strongly to inoculation with AMF. 
	
	\textit{Their work}. They want to see if they can use genetic methods to harness the best power of AMF. Two types of variation in AMF: variation among individuals and variation inside the fungus. If we could use genetic variation in beneficial microbes to improve crop yields, this would help us focus more on improving the microbes. If variation in microbes can significantly alter crop yields then this could potentially bring about a paradigm shift in how we try to increase agricultural productivity. AMF reproduces clonally. When new spores are produced by the fungus, they all receive the 2 genotypes of the 2 different nuclei (if it's a dikaryon), but they don't receive the same proportions. We don't know how this occurs - it just happens by chance. It causes this change in proportions of alleles. Question: will this lead to differences in plant growth? The answer is yes. You can generate a lot of variation in cassava growth just by inoculating the cassava with these siblings: you can get differences of 3kg of roots per plant! Putting these results in context: we can alter cassava yield by up to 30 tons per hectare just by inoculating with 2 sister fungi; the global average cassava yield is only 12.8 tons per hectare; agronomists are usually happy if a new management practice will change crop growth by 10-20 percent; adding phosphate fertilizer has almost no effect on yields in these soils. They also tested for whether these effects are really due to genetics by comparing to homokaryon parents' offspring: variation in plant growth is large but how much is really due to genetic differences among the fungal siblings? They found not strong enough support for a fungal genetic variation component: plants inoculated with homokaryon siblings vary as much in growth as those inoculated with dikaryons. If everything was due to the genetics of the fungus, you would expect variation to be high among plants inoculated with siblings of parents which are dikaryons versus low for homokaryons. But it's actually high for both: this is about the AMF epigenome - they're doing a separate strand of research into this. But: they do have evidence from other experiments that genetic variation in AMF leads to differences in cassava growth. Now what about AMF interactions with the plant and soil microbiome? Plants already have an AMF community in their roots: inoculating with AMF just adds a bit to the fungus already present. Plants channel 20-30 percent of fixed carbon into AMF which then goes out into the soil to feed the soil microbiome. AMF are responsible for creating soil structure and, thus, contributing to belowground carbon storage. So what happens to the soil structure when we add AMF? Found that alpha diversity (OTU richness) of native AMF was significantly altered by inoculation with sibling AMF, but only in one plant variety (Ordonez et al. unpublished). They found inoculation changes positive networks between mycorrhizal fungal taxa. So how does inoculation with AMF affect the bacterial microbiome? They sequenced the soil metagenome: found that diversity of genes in the bacterial metagenome is much higher when you add the fungus compared to when you don't. Different AMF also strongly affect gene richness and gene diversity of the soil bacterial metagenome. Different AMF do not affect the taxonomic beta diversity of the bacterial community but enormously shape gene composition: they found that the abundance of 65,000 genes was significantly altered in the bacterial metagenome between two different treatments. Next question: how does AMF addition influence soil structure and carbon storage, as well as how much carbon is emitted via microbial respiration of CO2? They found it did indeed affect soil carbon sequestration: 14 percent greater organic carbon storage in small aggregates in all inoculated treatments, and 35 percent difference in organic carbon in large aggregates. Found that AMF inoculation greatly altered microbial respiration in the soil (up to 7-fold). Inoculation therefore affects CO2 release from the soil. Hence this is all how such a small amount of AMF can affect cassava!
	
	 
	
	
	
	



	
	
	
	
	
	\bibliography{seminar_diary}
	
\end{document}