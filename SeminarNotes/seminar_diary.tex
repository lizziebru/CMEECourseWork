\documentclass[9pt]{article}
\usepackage[utf8]{inputenc}
\usepackage{setspace}
\usepackage{lineno}
\usepackage{authblk}
\usepackage{natbib}
\usepackage[margin=1cm]{geometry}
\usepackage{tabularx}
\newcommand{\supersc}[1]{\ensuremath{^{\textrm{#1}}}}
\newcommand{\subsc}[1]{\ensuremath{_{\textrm{#1}}}}
\usepackage{booktabs}
\bibliographystyle{unsrtnat}
\usepackage{graphicx}

\title{\textbf{Seminary diary}}
\author[1]{Lizzie Bru}
\affil[1]{School of Life Sciences, Imperial College London, Silwood Park Campus, Ascot SL5 7PY, UK}
\date{}

\onehalfspacing


\begin{document}
	
	\maketitle
	
	\newpage
	
	\section{\textit{Underground forests, fire, mammal browsers, and the origins of African savanna}. Professor Jonathan Davies, University of British Columbia}
	
	Major question: what allowed savanna to expand and separated it from other biomes?
	
	Background on biomes in South Africa: Fynbos are one of the many hyperdiverse biomes, being both very small and very diverse. Contrastingly to Fynbos, the Namib desert biome has very low diversity, but contains some of the tallest sand dunes in the world. The grassland biome resembles savanna landscapes but has a different community of flora. The Indian Ocean Coastal Belt biome is more forested. The Miombo woodland biome contains more thickets and shrub-like vegetation, and is the closest biome to the savanna biome. The Savanna biome is present across much of South Africa. It contains sporadic trees interspersed with grassland wherever they are not dense enough to competitively exclude grass. Savanna could support forests, however, so there must be something preventing forests from dominating this biome.
	
	Phylogenetic analysis: DNA barcoding of the Trees of Africa (initiative run by the African Centre for DNA barcoding) was used to build a phylogenetic history of South African flora. They sequenced anything that fills a tree-like role in the landscape, randing from shrub to bamboo to climbers. Daru et al. (2015) used the tree to generate a new map of the biomes of South Africa based on phylogenetic relationships. This showed that these biomes contain species that phylogenetically cluster together.
	
	Origin of these biomes: could this phylogenetic information be used to address the origin of these biomes? Professor Davies and his colleagues successively removed phylogenetic structure to identify the evolutionary depth at which the biomes collapse to see how the biomes collapse into each other. Their initial best guess for the age of savanna is 8-60 million years. While climate is the usual explanation for changing vegetation, it does not match the time of savanna spread. They found that there was a decrease in carbon dioxide levels and C4 grass lineages; while this does not on its own explain the origin of savannas, it was perhaps necessary. A kickstart was therefore needed for the ecological expansion into savanna. A possibility is fire which would have pushed back the forest. Grazers could also push back the forest. These two competing explanations are both good possibilities, but there is little evidence supporting either. We therefore have a very poor record of what the ultimate drivers of savanna origin were due to no information on browser or fire intensity from the fossil record. On possibility is studying the "underground forests" of Africa (\textit{Geoxylic suffrutex}): unique species where most of them are underground, but their sister lineages look like trees. They are only found in savanna grassland, so their age could be useful for dating savanna since they require savanna to exist. They could be used as markers for fire since their growth evolved in response to frequent fires and high precipitation. Therefore, perhaps fire was one of the ecological forces that allowed savanna expansion. They performed Bayesian reconstruction of the divergence times between two of these sister taxa (\textit{Cussonia arborea} and \textit{Cussonia corbisieri}). This showed that the radiation of underground trees happened multiple times independently. Despite large uncertainty with ages, the results showed that there are no very old underground trees, with most being 2-4 million years old and no evidence of old underground trees at all in the south. Perhaps they therefore originated at the equator and more recently evolved in the south, suggesting that fire drove the expansion of savannas, originating at the equator and driving south. 

	
	
	\section{\textit{The limits to ecological limits to diversification}. Professor Rampal Etienne, University of Groningen}
	
	Introduces the competition-related hypothesis, which hypothesises that more closely related species will share similar habitats and constitutions and thus experience higher competition and speciation rates than unrelated species do. There is evidence of limits to ecological diversification. Fossils of extinct and current diversity tell us that extinction happens/diversity does not always increase exponentially, providing evidence for biodiversity becoming limited at the species level/macroevolutionary scale. Molecular phylogenies of extant species show that diversity does not increase exponentially. Both fossils and phylogenies provide some evidence suggestive of negative feedback of diversity on diversification, suggesting that diversification is diversity-dependent. Possible biological mechanisms include: niches become increasingly occupied and eventually limit diversification; or niche construction slows down as species abundance increases (although in some cases it can go up too). The problem is that many phylogenies don't plateau or show an inverted S-shape. Possibly not enough time was allowed. The model may therefore be inaccurate (e.g. protracted speciation instead could explain the slow-downs, we could be over-simplifying adaptive radiation). This suggests that whether diversification depends on diversity depends on the species in question. If this is the case, this leads to the question of what are these limits to diversity-dependence i.e. the limits to ecological limits to diversification. Islands could be a good way to answer this question. 
	
	There are two types of diversity-dependence: clade-specific diversity-dependence, which only applies to diversity within the clade established by the mainland ancestor; and island-wide diversity-dependence, where all species on the island affect each other. They present a new method for distinguishing these two types of diversity-dependence. This involves using the stochastic framework Dynamic Assembly of Island biotas through Speciation, Immigration and Extinction (DAISIE). This allows diversity-dependent diversification and immigration, and parameters estimated using likelihood of phylogenetic data can be used for fully stochastic simulations. (Remember rates of colonisation and extinction depend on no. of species present on an island - theory of island biogeography). Simulation work showed that DAISIE is robust to model violations: specifically time-varying rates due to island ontogeny/environmental fluctuations, trait-dependent speciation, extinction on the mainland, mainland sampling, and non-oceanic island origin. They used DAISIE to estimate rates of extinction, using parametric bootstrapping. Using DAISIE on Hispaniola rain frogs, they found evidence for clade-specific but not island-wide diversity-dependence. The observed repeated patterns could suggest that niche differences exist between different clades. To summarise, they therefore found evidence for limits to ecological limits to diversification using tools to detect this from phylogenetic data, notably from islands. Going forwards, they therefore developed a method which is operational, although it is mathematically and computationally challenging, and there are not many datasets currently available for larger-scale hypothesis testing.


	\section{\textit{Birdwatching on a cosmic scale}. Dr Daniel J. Field, University of Cambridge}
	
	Dan's work is based on understanding how the diversity of birds has arisen. Over the last 10-15 years, this has become really important in ornithology, and major steps have been made in resolving evolutionary relationships among living groups of birds (e.g., \cite{prum2015comprehensive}). A particularly important point in time is the end Cretaceous mass extinction, when a major asteroid (11-81km in diameter, travelling at ~20km/s, causing an impact crater fo 150km in diameter and 20km deep) struck the Earth. It drove major groups of organisms to extinction, but also it's important to consider how this event may have paved the way for structuring survivorship/diversification patterns for those lineages that survived the event. We thus want to understand how this major event played a role in the evolutionary diversification of birds. 
	
	Gave a quick summary of how the K-Ph transition influenced bird evolution (from \cite{longrich2011mass} and other work by Dan mostly). Loads of major groups of organisms disappeared. We knew there were many lineages of these birds in the Cretaceous but didn't know when they went extinct. All of these major lineages extend through the fossil record up until this asteroid impact then they don't appear in the fossil record anymore, suggesting their extinction was probably driven by this impact. But we know that birds exist in the present day, so some of them must have survived. So Dan's research showed that birds suffered a lot in this mass extinction event, but remaining questions are: which birds did survive? How did they manage to survive? What happened to bird habitats?
	
	To answer the first question of which birds survived, they used a time-calibrated phylogeny. They think that only a small handful of crownbird lineages survived this mass extinction event (\cite{longrich2011mass}). To answer the second question about how these birds survived, they looked into body size because he thinks it was a highly size-selective event: hardly anything larger than 5kg survived. And they found evidence for this: relatively strong evidence that post-extinction crownbirds were smaller than pre-extinction ones (\cite{berv2018genomic}). For the question of what happened to bird habitats: they hypothesised that global deforestation occurred as a result of the impact inducing a selective filter against arboreal birds. They tested this by incorporating ecological data into large-scale ancestral state reconstruction. The Mesozoic fossil record is relatively rich for providing evidence of what pre-modern birds were like in the late part of the Cretaceous. They found that only ground-dwelling lineages survived the mass extinction event (\cite{field2018early}).
	
	Three key hypotheses about crownbird macro-evolution following this mass extinction event: deep evolutionary branch (only the deepest lineages survived), relatively small-bodied (any surviving lineages probably would've been relatively small-bodied), and non-tree-dwelling (any surviving lineages probably would've been mostly non-tree-dwelling). But these are only hypotheses: the only way to test them is to look at the crownbird fossil record. However, the earliest fossils of crownbirds in the Cretaceous is very scarce. They did find this one fossil encased in limestone: an unassuming holy grail (\cite{field2020late})! It's a fossil of a femur and tibiotarsus. Dates back to just before the asteroid impact. CT-scanned it and found it has a nearly complete 3D skull. Lower jaw is fused together, lots of features support the idea that it's a member of the Neornithes clade (early modern-type bird) - they gave it a new taxon name, Galliforme - , and phylogenetic analyses support this too. What does this fossil tell us about these hypotheses? Hypothesis 1: it's one of the 3 deepest lineages in the crownbirds - and it crossed the extinction event. Hypothesis 2: it's considerably smaller than many birds from that time period that were also ground-dwelling. Hypothesis 3: hind-limb proportions suggest it was ground-dwelling. So this fossil supports these hypotheses! Plus it's from Europe which has implications for future more European finds!


	\section{\textit{A call to ARMS as a Marine Biodiversity Observation Network}. Dr Christopher Meyer, Dr Sarah Tweedt, National Museum of Natural History, Smithsonian Institution}
	
	Chris: works on assessing the vast crypto-benthic community diversity in French Polynesia. Big question: how should we count and inventory? How (e.g. DNA) and where should we measure? This is how the ARMS project came about. They stack PVC plates laid on the seafloor then recover them. These are standardised everywhere, so they use these as a standard metric of biodiversity. They do size fractionation so that not only one creature dominates. Their data shows remarkable turnover across the Pacific Ocean. Using standardised transects, they found that you can only find one percent of all the species everywhere. How can they use ARMS to study how diversity responds to change? What are the tipping points for communities? They collected data across  gradients of degradation/recovery: anthropogenic gradients (more human impact == lower diversity), pH gradients (to look at effects of ocean acidification: found lower diversity when lower pH), depth gradients (== deeper spots have less diversity than shallower spots), protected vs unprotected areas (found that systems are able to recover pretty well - regain higher diversity within 2-3 years when area is put under protection). They also tested this with different marker choices: found that overall you get pretty similar results. They found that the year-to-year effect is more than the space-to-space effect. They want to expand to a global ARMS program, to make it a tool that anyone can use throughout the world. Also useful as an invasive species detector. It's now been adopted in the newly formed European Marinomics network, with 20 ARMS observatories, 14 European countries, and half of the sites being used to monitor for non-indigenous species. 
	
	Sarah: ARMS are great to sample marine sponges: they sample cecil-encrusting organisms. Sponges have had ~800 million years to evolve, and most arms plates collect loads of taxonomically-important sponges. eDNA (= DNA collected from environmental samples (soil, water etc))is really useful for ecosystem monitoring, species detection etc. And it's cheap and easy to implement. It's a hot topic at the moment, with government agencies asking how we can use eDNA to monitor things more cheaply etc. ARMS as a teabag/seepage experiment: can eDNA from ARMS-bin water samples recover the ARMS profile? Will sponges be the first organisms detected in eDNA from ARMS water? Relevant paper: \cite{de2009cell}.They chose 2 ARMS to soak in their seepage experiment. Sampling protocol: 2 Tetiaroa ARMS (B ad D); first sample = time 0 = ambient water from locality; collect ARMS water for 8 time points, 0-24 hours; water in bin gently mixed, 1.0 litres collected, 40 micrometre-filtered to remove plankton particles; later filtered through 0.22 micrometre sterivex (50ml syringe), preserved in ~2.0ml ATL. Extraction, amplification, and sequencing protocol: digestion and DNA extraction; COI amplification with index adaptors, indexing PCR; normalize, pool samples, MiSeq sequencing. Sequence analysis protocol: import raw sequences + trim primers; read quality control; generate 'ASV' table + analysis. Gave some details about each step (although gave a bit too much jargon-y detail on this for a talk). Cool result: they found that eDNA and ARMS separate nearer the bottom. Found lots on marine fungi. Also, two species of parrotfish (+ also diadema) spawned and every single sample is swamped with parrotfish and diadema DNA: this really shifted the eDNA profiles. More recent work: \cite{antich2021marine}. Most reads in the ARMS get picked up in the eDNA which is good. 
	
	
	
	\section{\textit{Genetic and epigenetic variation in mycorrhizal fungi and its role in altering plant growth and carbon storage in tropical soils}. Dr Ian Sanders, University of Lausanne}
	
	\textit{Introduction}. Malnutrition kills loads of people annually. What can scientists do about it? Cassava! It's very important for food security in the face of climate change. Cassava has a relatively stable yield during perturbation of climate (better than other crops). How can we tackle malnutrition with fungi? Arbuscular mycorrhizal fungi (hereafter AMF) helps plants grow bigger by helping plants get more nutrition from the soil. Lost of people work trying to improve cassava production: cassava responds very strongly to inoculation with AMF. 
	
	\textit{Their work}. They want to see if they can use genetic methods to harness the best power of AMF. Two types of variation in AMF: variation among individuals and variation inside the fungus. If we could use genetic variation in beneficial microbes to improve crop yields, this would help us focus more on improving the microbes. If variation in microbes can significantly alter crop yields then this could potentially bring about a paradigm shift in how we try to increase agricultural productivity. AMF reproduces clonally. When new spores are produced by the fungus, they all receive the 2 genotypes of the 2 different nuclei (if it's a dikaryon), but they don't receive the same proportions. We don't know how this occurs - it just happens by chance. It causes this change in proportions of alleles. Question: will this lead to differences in plant growth? The answer is yes. You can generate a lot of variation in cassava growth just by inoculating the cassava with these siblings: you can get differences of 3kg of roots per plant! Putting these results in context: we can alter cassava yield by up to 30 tons per hectare just by inoculating with 2 sister fungi; the global average cassava yield is only 12.8 tons per hectare; agronomists are usually happy if a new management practice will change crop growth by 10-20 percent; adding phosphate fertilizer has almost no effect on yields in these soils. They also tested for whether these effects are really due to genetics by comparing to homokaryon parents' offspring: variation in plant growth is large but how much is really due to genetic differences among the fungal siblings? They found not strong enough support for a fungal genetic variation component: plants inoculated with homokaryon siblings vary as much in growth as those inoculated with dikaryons. If everything was due to the genetics of the fungus, you would expect variation to be high among plants inoculated with siblings of parents which are dikaryons versus low for homokaryons. But it's actually high for both: this is about the AMF epigenome - they're doing a separate strand of research into this. But: they do have evidence from other experiments that genetic variation in AMF leads to differences in cassava growth. Now what about AMF interactions with the plant and soil microbiome? Plants already have an AMF community in their roots: inoculating with AMF just adds a bit to the fungus already present. Plants channel 20-30 percent of fixed carbon into AMF which then goes out into the soil to feed the soil microbiome. AMF are responsible for creating soil structure and, thus, contributing to belowground carbon storage. So what happens to the soil structure when we add AMF? Found that alpha diversity (OTU richness) of native AMF was significantly altered by inoculation with sibling AMF, but only in one plant variety (Ordonez et al. unpublished). They found inoculation changes positive networks between mycorrhizal fungal taxa. So how does inoculation with AMF affect the bacterial microbiome? They sequenced the soil metagenome: found that diversity of genes in the bacterial metagenome is much higher when you add the fungus compared to when you don't. Different AMF also strongly affect gene richness and gene diversity of the soil bacterial metagenome. Different AMF do not affect the taxonomic beta diversity of the bacterial community but enormously shape gene composition: they found that the abundance of 65,000 genes was significantly altered in the bacterial metagenome between two different treatments. Next question: how does AMF addition influence soil structure and carbon storage, as well as how much carbon is emitted via microbial respiration of CO2? They found it did indeed affect soil carbon sequestration: 14 percent greater organic carbon storage in small aggregates in all inoculated treatments, and 35 percent difference in organic carbon in large aggregates. Found that AMF inoculation greatly altered microbial respiration in the soil (up to 7-fold). Inoculation therefore affects CO2 release from the soil. Hence this is all how such a small amount of AMF can affect cassava!
	
	 
	
	\section{\textit{Using 30 years of UK monitoring data to identify the impacts of chemicals on invertebrates}. Professor Andrew Johnson, UK Centre of Ecology and Hydrology}
	
	There are a large amount of data on invertebrates which have not really been used yet to investigate the effects of chemical pressures. Chemicals in the environment: especially recently, there has been much interest in declining water quality, partly due to low poor quality indicator thresholds, leading to the belief that rivers are very polluted and this is causing problems for wildlife. He addresses the question: is this true for UK rivers? If we examine wildlife monitoring records with respect to chemical exposure over the past 30 years, what can we learn from England? He’ll address this by focussing on macroinvertebrates. 
	
	\textit{Building a case against a chemical}. Chemical-focussed approach (work out toxicity thresholds in lab; typically used in Europe); unique wildlife failure approach (examine what caused the failure e.g. sudden population crash of Asian vultures); unique location failure (examine what chemicals caused problems there). The chemical-focussed approach can be both prospective and retrospective (therefore is the most common); the others are just retrospective. However, it can underpredict (e.g. due to additive mixture effects) or overpredict effects (e.g. due to lack of ecological realism). His focus is using field data (to him: more important than laboratory data) to help better understand the harm of chemicals. We can use some of Austin Bradford Hill’s 1995 epidemiology criteria using field data: e.g. consistency: are patterns temporally and spatially invariable? There are some 15,000 macroinvertebrate data locations and 3,5000 fish data locations from the past 10-30 years, providing millions of records!
	
	\textit{His project: methods}. They use the Low Flows 2000-WQX model to estimate wastewater exposure of any reach of river in the UK. This is based on the LF2000 hydrological model for estimating river flows at ungauged sites. Wastewater is likely to be the most major source of chemicals that wildlife is exposed to as it is made up of a cocktail of many chemicals e.g. organics, pharmaceuticals. So it’s a useful metric for wildlife exposure to chemicals. The model predicted percentage wastewater effluent in rivers across England and Wales. E.g. in the Midlands, around 1/3 of locations have over 25 percent wastewater, making them amongst the river networks most exposed to wastewater in Europe. Some trends in chemical determinands observed include e.g. in the midlands found a declining trend in ammonia. In order to assign a metric to an assemblage of macroinvertebrates in a riverbed, they give an aggregate score to represent all the invertebrates present. Each taxon is given a score on its pollution tolerance, they calculate species richness, and combine all of this to get the score. 
	
	\textit{Results on biodiversity trends for high and low wastewater sites across England over the last 30 plus years}. Increase in species richness and other richness indices for both high and low wastewater sites (both urban and rural). Generally you get a higher score in the lower wastewater sites, but trends are more interesting. The improvement in high wastewater sites could be plausibly linked to the introduction of the urban wastewater directive in 1991 (requirement of higher wastewater treatment). Results with respect to geography: there has been an increase in the number of sites with high macroinvertebrate species richness throughout the country. This could be linked to the introduction of the National Rivers Agency in 1991. Question: which chemicals matter and which don’t? Still examining this – models including GLMM, GAMM, PCA-GAMM and Bayesian. The models suggest different things but BOD, DO, NH3, and PO4 are strongly featured. Conducted similar analyses for invertebrates on land: is there a problem with invertebrates associated with agricultural practice over the past 29 years? Found that they are declining in biodiversity, especially in arable land use areas. We are therefore currently managing our surface waters well and our terrestrial environment poorly. 
	
	
	\section{\textit{Making Space for Nature: the science underpinning the Lawton report}. Sir John Lawton - NEED TO LOOK FOR REFERENCES IN TEXT!!}
	
	\textit{About the report}. Secretary of State in Defra (Hilary Benn) asked him in 2009 to chair a review of the state of England’s protected area network. Prompted by data on the parlous state of many UK organism populations. Aims: examine evidence on the extent to which England’s collection of wildlife sites represents a coherent and resilient ecological network capable of adapting to challenges e.g. climate change; examine evidence base to assess whether we need a more inter-connected network and how we could do this; consider ecological, economic, and social costs and benefits; and make recommendations. 
	
	\textit{Main conclusions and underlying biology}. Identified three tiers of wildlife sites making up the network: tier 1 (primarily for nature conservation, high level of protection e.g. Local Nature Reserves, make up 6.9 percent of England’s land-area, including freshwater sites); tier 2 (designated for high biodiversity value but don’t receive full statutory protection, e.g. Local Wildlife Sites, make up 6.5 percent of UK land); tier 3 (primarily designated for other reasons but wildlife conservation is included in statutory purpose, e.g. AONBs (14.4 percent), National Parks (9.1 percent)). Many areas don’t have a designation, and the government often double count (e.g. areas in multiple tiers). The report concluded that England’s wildlife sites don’t comprise a coherent and resilient ecological network (e.g. many too small, not well-managed). The solution: more, bigger, better, and joined, and where possible reduce pressure on wildlife by improving surrounding buffering areas, doing this at a scale that creates a step-change in conservation gains. History of the Yorkshire Wildlife Trust illustrates the history of conservation movements in the UK: steadily adding more reserves - but more isn't sufficient. Why: species-area relationships (need larger areas for more species); fewer bigger areas are cheaper to manage; smaller populations are more likely to fluctuate to extinction; big reserves usually have more varied topography, allowing higher species richness and resilience to climate change. There's also been 80 percent loss of lowland heath in England since 1800 – habitat management and restoration is now reversing these losses (more on this in seminar 2). We also generally need “better” reserves: most UK reserves aren't 'natural' – very old but still human-affected, e.g. grassland – need management to maintain them, but we're often not managing them well. There’s also an important interaction between “bigger” and “better”: Armsworth et al. 2010 found it’s much cheaper to manage habitats in large reserves than small reserves (see Armsworth et al. 2010: Management costs for small protected areas and economies of scale in habitat conservation – find reference for this). “Joined” is also important so that species can move and migrate: the more isolated, the fewer species per given area (e.g. Gaston and Blackburn 2000 – find reference too).  
	
	\textit{Policy responses}. Making Space for Nature was adopted by the government remarkably quickly. E.g. National competition to create 12 Nature Improvement Areas: first government-funded habitat restoration scheme. Response to report's recommendation to instal ecological restoration zones. E.g. one in Birmingham and Black Country was an urban landscape: around 25km2 of new habitat with additional benefits e.g. improved connectivity. The government made specific commitments to recognise the poor state of current reserves, to increase restoration etc.. But like many government initiatives they did not translate their commitments into law, it was unclear where the money would come from, etc.. In the report, they also made recommendations on how to carry out these actions alongside delivering societal benefits e.g. paying farmers to deliver benefits. January 6th 2022: government unveiled plans to deliver several of these recommendations. E.g. the Local Nature Recovery Scheme – creating habitats e.g. ponds, wetlands in farms (land sharing). But there are problems with the government’s policies e.g. they’ve resisted attempts to put a clear and firm duty on Local Authorities to create the Local Nature Recovery Strategy. However, it’s still a step in the right direction. It shows how policy formulation is a messy, iterative, slow process involving economics, cultural values, tensions between institutions, different interpretations of science, and the need to score political points. There are many reasons why politicians and policy-makers fail to act on ‘expert’ scientific advice: e.g. scientists not explaining themselves enough (so-called ‘simple deficit model’), so politicians use the uncertainty as a reason to do nothing – although this is only to a limited extent; also due to e.g. institutional failure, corrupt politicians etc. Yet if things line up responses can be rapid. E.g. Secretary of State Caroline Spelman and her minister Richard Benyon genuinely cared so NIA setting up was successful. There are things that can help: e.g. presenting the science in a positive good news story – makes it more palatable to politicians and seems to make them more likely to take action. There is some action to deliver the Lawton initiatives, but the majority of change is coming from individual landowners etc – they’re crucial for delivering these initiatives.
	
	
	\section{\textit{The Lawton report in the historical context of wildlife conservation in the UK - past, present, and future}. Sir John Lawton}
	
	
	\textit{Past}. Establishing Nature Reserves in the UK started in the early part of the twentieth century, with three later phases following this. Phase 1: preservation by the establishment of protected areas of ‘good’ habitats (nature reserves). Phase 2: deliberate habitat creation within existing reserves. Phase 3: deliberate habitat creation outside existing reserves to increase size and number of protected areas. Phase 4: (re-)wilding. ‘Modern’ nature conservation in the UK was started by Charles Rothschild, and entomologist. He convened a meeting at the British Museum with three friends (Charles Edward Fagan, William Ogilvie-Grant, and Francis Henley) to formulate the first national strategy for nature conservation in the UK. He realised that entire habitats need protection in addition to species. We can’t say precisely when the first nature reserves were established, but Wicken Fen and Woodwalton Fen are two of the oldest. 
	
	\textit{Present}. Sadly, over the last 100 years many sites identified by Rothschild have been destroyed. Of the 182, only 19 survive essentially intact today. The 3rd State of Nature Report (2019) shows that the situation is definitely bad: e.g., the report said that the UK would miss most of our biodiversity targets for 2020: which we did. Pollution, urbanisation, construction of infrastructure etc are causing big problems for habitat patches and populations. But changes in agriculture over the last 150 years have had the biggest effects on UK wildlife. Why is this the case? Agriculture was in serious depression from 1870-1940, leaving much land uncultivated. 1950-1970: Ian Newton describes this as the “post-war farming revolution”, involving mechanisation etc. From 2000, agriculture became extremely hostile for wildlife e.g. pesticide use. So although the number of protected sites increased in the second half of the 20th century, there became much less space for nature due to agriculture, causing species declines. So we haven’t got enough nature in our existing sites, so we need to restore nature rather than simply preserve it. E.g. first RSPB project in restoration was the creation of the Lakenheath Fen in Suffolk. E.g. Nature Improvement Areas from seminar 1 too. These examples are land sharing though – there’s also land sparing. Ongoing debate about which is better. Conservation in the UK is now moving towards habitat restoration and re-creation. National Trust and lots of other landowners very important in this e.g. National Trust Priority Habitats initiative, also lots of others e.g. Wildlife Trusts Living Landscapes. 
	
	\textit{Re-wilding and an eye to the future}. Re-wilding: there’s no generally agreed decinition. His thinking is scale is not the main distinction between re-wildling and land sharing. Re-wilding, unlike land-sharing, is not goal oriented: it lets natural ecosystem processes be restored naturally. Alistair Driver (director of Rewilding Britain) definition: “the large-scale restoration of ecosystems to the point where nature can take care of itself”. But some people are scared by the extremes of re-wilding e.g. re-introducing wolves into Britain. There are some mixed models of re-wilding, land sharing, and nature reserves e.g. Wild Ken Hill in Norfolk. Often wider societal benefits e.g. Ennerdale: re-wilding provided better water supplies and flood control. 
	
	Carry on from: 41:08
	
	
	\section{\textit{The Lawton report: the challenges in measuring success in efforts to restore nature}. Sir John Lawton}
	
	
	
	



	
	
	
	
	
	\bibliography{seminar_diary}
	
\end{document}